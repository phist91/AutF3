%!TEX root = AutF3.tex
% -- Author: Phil Steinhorst, p.st@wwu.de
\documentclass[a4paper,index=totoc,toc=bibliography,fontsize=11,DIV=12,headinclude,BCOR=12mm,cleardoublepage=empty]{scrreprt}

\usepackage[usenames,x11names]{xcolor}
\usepackage[final]{graphicx}
\usepackage{lmodern}
\usepackage{mathtools,amssymb,amsthm}
\mathtoolsset{showonlyrefs}
\usepackage[libertine,cmintegrals,bigdelims,varbb]{newtxmath}
\usepackage[no-math]{fontspec}
\usepackage{polyglossia}
\setmainlanguage[spelling=new,babelshorthands=true,latesthyphen]{german}
\shorthandoff{"}
\defaultfontfeatures{Mapping=tex-text, Ligatures={Required,Common,Contextual}}
\setmainfont{LinLibertine}[Extension=.otf,UprightFont=*_R,BoldFont=*_RZ,ItalicFont=*_RI,BoldItalicFont=*_RZI]
\setsansfont{LinBiolinum}[Scale=MatchUppercase, Extension=.otf, UprightFont=*_R, BoldFont=*_RB, ItalicFont=*_RI,BoldItalicFont=*_RBO]
\setmonofont{Inconsolatazi4}[Scale=MatchUppercase,Extension=.otf,UprightFont=*-Regular,BoldFont=*-Bold,StylisticSet=1]
\usepackage[final]{microtype}
\mathtoolsset{centercolon}
\usepackage{mathdots}

\usepackage{tikz-cd}
\tikzset{
  every picture/.append style={
    execute at begin picture={\shorthandoff{"}},
    execute at end picture={\shorthandon{"}}
  }
}
\usetikzlibrary{quotes,babel}
\usetikzlibrary{patterns}
\tikzset{
	schraffiert/.style={pattern=north west lines,pattern color=#1},
	schraffiert/.default=black
}

\usepackage[
	backend=biber,
	sortlocale=auto,
	natbib,
	hyperref,
	style=alphabetic
	]%
{biblatex}
\setlength{\bibitemsep}{1.5em}  
\setlength{\bibhang}{2em}    
\addbibresource{literature.bib}

\usepackage[hidelinks, pdfpagelabels, bookmarksopen=true, bookmarksnumbered=true, linkcolor=black, urlcolor=SkyBlue2, plainpages=false,pagebackref, citecolor=black, hypertexnames=true, pdfauthor={Phil Steinhorst}, pdfborderstyle={/S/U}, linkbordercolor=SkyBlue2, colorlinks=false,backref=false]{hyperref}
\hypersetup{final}

\usepackage[shortlabels]{enumitem}
\setlist[enumerate,description]{font=\sffamily\bfseries}
\usepackage[german=quotes]{csquotes}

\usepackage[textsize=small]{todonotes}
\setlength{\parindent}{0em} 

\usepackage{scrpage2}
\pagestyle{scrheadings}
\clearscrheadfoot 
\setheadsepline{1pt} 
\automark[chapter]{chapter}
\rohead{\rightmark} 
\lehead{\rightmark} 
\ofoot[\pagemark]{\pagemark} 

\usepackage{thmtools}

\declaretheoremstyle[%
	headfont=\sffamily\bfseries,
	notefont=\normalfont\sffamily,
	bodyfont=\normalfont,
	headformat=\NAME \ \NUMBER \NOTE,
	headpunct={\\},
	postheadspace=1ex,
	spaceabove=15pt,spacebelow=10pt]%
{mainstyle}
\declaretheoremstyle[%
	headfont=\sffamily\bfseries,
	notefont=\normalfont\sffamily,
	bodyfont=\normalfont,
	headformat=\NAME \ \NOTE,
	headpunct={\\},
	postheadspace=1ex,
	spaceabove=15pt,spacebelow=10pt]%
{miscstyle}
\declaretheoremstyle[%
	headfont=\bfseries\scshape,
	bodyfont=\normalfont,
	headpunct=:,
	postheadspace=1ex,
	spacebelow=12pt,spaceabove=2pt,
	qed=\qedsymbol]%
{beweise}
\declaretheoremstyle[%
	headfont=\sffamily\bfseries,
	notefont=\normalfont\sffamily,
	bodyfont=\normalfont,
	headformat=\NAME,
	headpunct={\\},
	postheadspace=1ex,
	spaceabove=15pt,spacebelow=10pt]%
{prestyle}

\declaretheorem[name=Definition,parent=chapter,style=mainstyle]{definition}
\declaretheorem[name=Satz,sharenumber=definition,style=mainstyle]{satz}
\declaretheorem[name=Korollar,sharenumber=definition,style=mainstyle]{korollar}
\declaretheorem[name=Lemma,sharenumber=definition,style=mainstyle]{lemma}
\declaretheorem[name=Proposition,sharenumber=definition,style=mainstyle]{proposition}
\declaretheorem[name=Beweis,numbered=no,style=beweise]{beweis}
\declaretheorem[name=Beispiel,numbered=no,style=miscstyle]{beispiel}
\declaretheorem[name=Bemerkung,numbered=no,style=miscstyle]{bemerkung}
\declaretheorem[name=Theorem,numbered=yes,style=mainstyle]{theo}
\declaretheorem[name=Definition,numbered=no,style=prestyle]{defpre}

\renewcommand\thetheo{\Alph{theo}}

% -- Author: Phil Steinhorst, p.st@wwu.de
\newcommand{\aff}{\mathbb{A}}
\newcommand{\CC}{\mathbb{C}}
\newcommand{\FF}{\mathbb{F}}
\newcommand{\HH}{\mathbb{H}}
\newcommand{\KK}{\mathbb{K}}
\newcommand{\NN}{\mathbb{N}}
\newcommand{\OO}{\mathbb{O}}
\newcommand{\QQ}{\mathbb{Q}}
\newcommand{\RR}{\mathbb{R}}
\newcommand{\ZZ}{\mathbb{Z}}
\newcommand{\bigO}{\mathcal{O}}					% Landau-O
\newcommand{\ind}{1\hspace{-0,8ex}1} 			% Indikatorfunktion (Doppeleins)

\newcommand{\ab}[1]{\overline{#1}}					% Abschluss
\newcommand{\bewrueck}{\glqq$\Leftarrow$\grqq:} 	% Beweis R�ckrichtung
\newcommand{\bewhin}{\glqq$\Rightarrow$\grqq:}		% Beweis Hinrichtung
\newcommand{\borel}{\mathfrak{B}}					% Borelsche Sigma-Algebra
\newcommand{\setone}{\{1\}}							% Einsmenge
\newcommand{\leb}{\lambda \hspace{-0,95ex}\lambda}	% Lebesgue-Ma� (Doppel-Lambda)
\newcommand{\Lp}{\mathcal{L}}						% L^p-R�ume
\newcommand{\NT}{\trianglelefteq}					% Normalteiler
\newcommand{\setnull}{\{0\}}						% Nullmenge
\newcommand{\weak}{\rightharpoonup}					% schwache Konvergenz
\newcommand{\weaks}{\overset{*}{\rightharpoonup}}	% schwache *-Konvergenz
\newcommand{\salg}{\mathfrak{A}}					% Sigma-Algebra (Skript-A)
\newcommand{\zyklot}[1]{#1^{(\infty)}}				% zyklotomische Erweiterung

\DeclareMathOperator{\Alt}{Alt} 					% Alternierende n-Linearform
\DeclareMathOperator{\Aut}{Aut} 					% Automorphismen
\DeclareMathOperator{\Bil}{Bil} 					% Bilinearformen
\DeclareMathOperator{\bild}{Bild} 					% Bild
\DeclareMathOperator{\Char}{char} 
\DeclareMathOperator{\dom}{dom} 					% Domain
\DeclareMathOperator{\diam}{diam}					% Durchmesser
\DeclareMathOperator{\dist}{dist} 					% Distanz
\DeclareMathOperator{\eqs}{\mathrel{\widehat{=}}}	% entspricht
\DeclareMathOperator{\diver}{div} 					% Gradient
\DeclareMathOperator{\EPK}{EPK} 					% Einpunktkompaktifizierung
\DeclareMathOperator{\End}{End} 					% Endomorphismen
\DeclareMathOperator{\esssup}{esssup}				% essentielles Supremum
\DeclareMathOperator{\Gal}{Gal}	 					% Galoisgruppe
\DeclareMathOperator{\ggT}{ggT} 					% ggT
\DeclareMathOperator{\GL}{GL}						% allgemeine lineare Gruppe
\DeclareMathOperator{\grad}{grad} 					% Gradient
\DeclareMathOperator{\Grad}{Grad} 					% Grad
\DeclareMathOperator{\Hess}{Hess} 					% Hesse-Matrix
\DeclareMathOperator{\Hom}{Hom} 					% Homomorphismen
\DeclareMathOperator{\id}{id} 						% identische Abbildung
\DeclareMathOperator{\im}{im} 						% image
\DeclareMathOperator{\Jac}{Jac} 					% Jacobson-Radikal
\DeclareMathOperator{\Kern}{Kern}					% Kern
\DeclareMathOperator{\kgV}{kgV} 					% kgV
\DeclareMathOperator{\Koker}{Koker} 				% Kokern
\DeclareMathOperator{\Cov}{Cov} 					% Kovarianz
\DeclareMathOperator{\Mod}{Mod} 					% Moduln
\DeclareMathOperator{\modu}{mod} 					% Modulo
\DeclareMathOperator{\ord}{ord} 					% Ordnung
\DeclareMathOperator{\der}{\partial}				% Partielle Ableitung
\DeclareMathOperator{\pot}{\mathcal{P}}				% Potenzmenge
\DeclareMathOperator{\prlim}{\varprojlim\limits}	% projektiver Limes
\DeclareMathOperator{\Quot}{Quot}					% Quotientenring
\DeclareMathOperator{\Rang}{Rang} 					% Rang
\DeclareMathOperator{\rot}{rot} 					% Rotation
\DeclareMathOperator{\sgn}{sgn} 					% Signum
\DeclareMathOperator{\Spec}{Spec} 					% Spektrum
\DeclareMathOperator{\SL}{SL} 						% Spezielle lineare Gruppe
\DeclareMathOperator{\SO}{SO} 						% Spezielle orthogonale Gruppe
\DeclareMathOperator{\SU}{SU} 						% Spezielle unit�re Gruppe
\DeclareMathOperator{\Spur}{Spur} 					% Spur
\DeclareMathOperator{\supp}{supp} 					% Tr�ger
\DeclareMathOperator{\Sym}{Sym} 					% Symmetrische Gruppe
\DeclareMathOperator{\tr}{tr} 						% trace

\DeclarePairedDelimiter{\abs}{\lvert}{\rvert}		% Betrag
\DeclarePairedDelimiter{\ceil}{\lceil}{\rceil}		% aufrunden
\DeclarePairedDelimiter{\floor}{\lfloor}{\rfloor}	% aufrunden
\DeclarePairedDelimiter{\sprod}{\langle}{\rangle}	% spitze Klammern
\DeclarePairedDelimiter{\enbrace}{(}{)}				% runde Klammern
\DeclarePairedDelimiter{\benbrace}{\lbrack}{\rbrack}% eckige Klammern
\DeclarePairedDelimiter{\penbrace}{\{}{\}}			% geschweifte Klammern

\DeclarePairedDelimiter\doppelstrich{\Vert}{\Vert}
\newcommand{\norm}[2][\relax]{
\ifx#1\relax \ensuremath{\doppelstrich*{#2}}
\else \ensuremath{\doppelstrich*{#2}_{#1}}
\fi}

\DeclareMathOperator{\lk}{lk}
\DeclareMathOperator{\cat}{CAT(0)}
\DeclareMathOperator{\Isom}{Isom}
\DeclareMathOperator{\prT}{(T)}
\DeclareMathOperator{\FSC}{F^sC_*}

\usepackage{setspace}
\usepackage{hyphenat}
\usepackage{bm}
\usepackage{titlesec}
\titlespacing*{\section}{0pt}{5.5ex plus 1ex minus .2ex}{2.2ex plus .2ex}
\usepackage{faktor}

\raggedbottom