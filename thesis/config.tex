%!TEX root = AutF3.tex
% -- Author: Phil Steinhorst, p.st@wwu.de
\documentclass[a4paper,index=totoc,toc=bibliography,fontsize=11,DIV=12,headinclude,BCOR=12mm,cleardoublepage=empty]{scrreprt}

\usepackage[usenames,x11names]{xcolor}
\usepackage[final]{graphicx}
\usepackage{lmodern}
\usepackage{mathtools,amssymb,amsthm}
\mathtoolsset{showonlyrefs}
\usepackage[libertine,cmintegrals,bigdelims,varbb]{newtxmath}
\usepackage[no-math]{fontspec}
\usepackage{polyglossia}
\setmainlanguage[spelling=new,babelshorthands=true,latesthyphen]{german}
\shorthandoff{"}
\defaultfontfeatures{Mapping=tex-text, Ligatures={Required,Common,Contextual}}
\setmainfont{LinLibertine}[Extension=.otf,UprightFont=*_R,BoldFont=*_RZ,ItalicFont=*_RI,BoldItalicFont=*_RZI]
\setsansfont{LinBiolinum}[Scale=MatchUppercase, Extension=.otf, UprightFont=*_R, BoldFont=*_RB, ItalicFont=*_RI,BoldItalicFont=*_RBO]
\setmonofont{Inconsolatazi4}[Scale=MatchUppercase,Extension=.otf,UprightFont=*-Regular,BoldFont=*-Bold,StylisticSet=1]
\usepackage[final]{microtype}
\mathtoolsset{centercolon}
\usepackage{mathdots}

\usepackage{tikz-cd}
\tikzset{
  every picture/.append style={
    execute at begin picture={\shorthandoff{"}},
    execute at end picture={\shorthandon{"}}
  }
}
\usetikzlibrary{quotes,babel}
\usetikzlibrary{patterns}
\tikzset{
	schraffiert/.style={pattern=north west lines,pattern color=#1},
	schraffiert/.default=black
}

\usepackage[
	backend=biber,
	sortlocale=auto,
	natbib,
	hyperref,
	style=alphabetic
	]%
{biblatex}
\setlength{\bibitemsep}{1.5em}  
\setlength{\bibhang}{2em}    
\addbibresource{literature.bib}

\usepackage[hidelinks, pdfpagelabels, bookmarksopen=true, bookmarksnumbered=true, linkcolor=black, urlcolor=SkyBlue2, plainpages=false,pagebackref, citecolor=black, hypertexnames=true, pdfauthor={Phil Steinhorst}, pdfborderstyle={/S/U}, linkbordercolor=SkyBlue2, colorlinks=false,backref=false]{hyperref}
\hypersetup{final}

\usepackage[shortlabels]{enumitem}
\setlist[enumerate,description]{font=\sffamily\bfseries}
\usepackage[german=quotes]{csquotes}

\usepackage[textsize=small]{todonotes}
\setlength{\parindent}{0em} 

\usepackage{scrpage2}
\pagestyle{scrheadings}
\clearscrheadfoot 
\setheadsepline{1pt} 
\automark[chapter]{chapter}
\rohead{\rightmark} 
\lehead{\rightmark} 
\ofoot[\pagemark]{\pagemark} 

\usepackage{thmtools}

\declaretheoremstyle[%
	headfont=\sffamily\bfseries,
	notefont=\normalfont\sffamily,
	bodyfont=\normalfont,
	headformat=\NAME \ \NUMBER \NOTE,
	headpunct={\\},
	postheadspace=1ex,
	spaceabove=15pt,spacebelow=10pt]%
{mainstyle}
\declaretheoremstyle[%
	headfont=\sffamily\bfseries,
	notefont=\normalfont\sffamily,
	bodyfont=\normalfont,
	headformat=\NAME \ \NOTE,
	headpunct={\\},
	postheadspace=1ex,
	spaceabove=15pt,spacebelow=10pt]%
{miscstyle}
\declaretheoremstyle[%
	headfont=\bfseries\scshape,
	bodyfont=\normalfont,
	headpunct=:,
	postheadspace=1ex,
	spacebelow=12pt,spaceabove=2pt,
	qed=\qedsymbol]%
{beweise}
\declaretheoremstyle[%
	headfont=\sffamily\bfseries,
	notefont=\normalfont\sffamily,
	bodyfont=\normalfont,
	headformat=\NAME,
	headpunct={\\},
	postheadspace=1ex,
	spaceabove=15pt,spacebelow=10pt]%
{prestyle}

\declaretheorem[name=Definition,parent=chapter,style=mainstyle]{definition}
\declaretheorem[name=Satz,sharenumber=definition,style=mainstyle]{satz}
\declaretheorem[name=Korollar,sharenumber=definition,style=mainstyle]{korollar}
\declaretheorem[name=Lemma,sharenumber=definition,style=mainstyle]{lemma}
\declaretheorem[name=Proposition,sharenumber=definition,style=mainstyle]{proposition}
\declaretheorem[name=Beweis,numbered=no,style=beweise]{beweis}
\declaretheorem[name=Beispiel,numbered=no,style=miscstyle]{beispiel}
\declaretheorem[name=Bemerkung,numbered=no,style=miscstyle]{bemerkung}
\declaretheorem[name=Theorem,numbered=yes,style=mainstyle]{theo}
\declaretheorem[name=Definition,numbered=no,style=prestyle]{defpre}

\renewcommand\thetheo{\Alph{theo}}

%!TEX root = AutF3.tex
% -- Author: Phil Steinhorst, p.st@wwu.de

\newcommand{\CC}{\mathbb{C}}
\newcommand{\HH}{\mathcal{H}}
\newcommand{\MM}{\mathcal{M}}
\newcommand{\NN}{\mathbb{N}}
\newcommand{\RR}{\mathbb{R}}
\newcommand{\UU}{\mathcal{U}}
\newcommand{\ZZ}{\mathbb{Z}}

\newcommand{\bewrueck}{\glqq$\Leftarrow$\grqq:} 	% Beweis R�ckrichtung
\newcommand{\bewhin}{\glqq$\Rightarrow$\grqq:}		% Beweis Hinrichtung
\newcommand{\setone}{\{1\}}							% Einsmenge
\newcommand{\NT}{\trianglelefteq}					% Normalteiler
\newcommand{\setzero}{\{0\}}						% Nullmenge

\DeclareMathOperator{\Aut}{Aut} 					% Automorphismen
\DeclareMathOperator{\GL}{GL}						% allgemeine lineare Gruppe
\DeclareMathOperator{\id}{id} 						% Identit�t
\DeclareMathOperator{\im}{im} 						% image
\DeclareMathOperator{\Isom}{Isom}					% Isometrien
\DeclareMathOperator{\Sym}{Sym} 					% Symmetrische Gruppe

\DeclarePairedDelimiter{\abs}{\lvert}{\rvert}			% Betrag
\DeclarePairedDelimiter{\ceil}{\lceil}{\rceil}			% aufrunden
\DeclarePairedDelimiter{\floor}{\lfloor}{\rfloor}		% aufrunden
\DeclarePairedDelimiter{\Norm}{\lVert}{\rVert}			% Norm
\DeclarePairedDelimiter{\sprod}{\langle}{\rangle}		% spitze Klammern
\DeclarePairedDelimiter{\enbrace}{(}{)}					% runde Klammern
\DeclarePairedDelimiter{\benbrace}{\lbrack}{\rbrack}	% eckige Klammern
\DeclarePairedDelimiter{\penbrace}{\{}{\}}				% geschweifte Klammern
\newcommand{\Underbrace}[2]{{\underbrace{#1}_{#2}}} 	% Underbrace als Befehl in LaTeX-Syntax (und ohne Spacing-Probleme mit nachfolgenden Operatoren...)

\DeclareMathOperator{\prT}{(T)}		% Eigenschaft (T)
\DeclareMathOperator{\prFH}{(FH)}	% Eigenschaft (FH)
\DeclareMathOperator{\prFC}{(FC)}	% Eigenschaft (FC)
\DeclareMathOperator{\prFA}{(FA)}	% Eigenschaft (FA)
\newcommand{\subfi}{\sqsubseteq}	% Untergruppe mit endlichem Index
\newcommand{\gind}[1]{\benbrace*{#1}}	% Gruppenindex
\DeclareMathOperator{\cat}{CAT(0)}
\DeclareMathOperator{\dom}{dom}
\DeclareMathOperator{\Fix}{Fix}
\DeclareMathOperator{\Eig}{Eig}
\DeclareMathOperator{\rk}{rk}
\DeclareMathOperator{\St}{St}
\DeclareMathOperator{\Inn}{Inn}
\DeclareMathOperator{\Ind}{Ind}

\usepackage{setspace}
\usepackage{hyphenat}
\usepackage{bm}
\usepackage{titlesec}
\titlespacing*{\section}{0pt}{5.5ex plus 1ex minus .2ex}{2.2ex plus .2ex}
\usepackage{faktor}

\raggedbottom