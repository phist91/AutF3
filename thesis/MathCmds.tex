%!TEX root = AutF3.tex
% -- Author: Phil Steinhorst, p.st@wwu.de

\newcommand{\CC}{\mathbb{C}}
\newcommand{\HH}{\mathcal{H}}
\newcommand{\MM}{\mathcal{M}}
\newcommand{\NN}{\mathbb{N}}
\newcommand{\RR}{\mathbb{R}}
\newcommand{\UU}{\mathcal{U}}
\newcommand{\ZZ}{\mathbb{Z}}

\newcommand{\bewrueck}{\glqq$\Leftarrow$\grqq:} 	% Beweis R�ckrichtung
\newcommand{\bewhin}{\glqq$\Rightarrow$\grqq:}		% Beweis Hinrichtung
\newcommand{\setone}{\{1\}}							% Einsmenge
\newcommand{\NT}{\trianglelefteq}					% Normalteiler
\newcommand{\setzero}{\{0\}}						% Nullmenge

\DeclareMathOperator{\Aut}{Aut} 					% Automorphismen
\DeclareMathOperator{\GL}{GL}						% allgemeine lineare Gruppe
\DeclareMathOperator{\id}{id} 						% Identit�t
\DeclareMathOperator{\im}{im} 						% image
\DeclareMathOperator{\Isom}{Isom}					% Isometrien
\DeclareMathOperator{\Sym}{Sym} 					% Symmetrische Gruppe

\DeclarePairedDelimiter{\abs}{\lvert}{\rvert}			% Betrag
\DeclarePairedDelimiter{\ceil}{\lceil}{\rceil}			% aufrunden
\DeclarePairedDelimiter{\floor}{\lfloor}{\rfloor}		% aufrunden
\DeclarePairedDelimiter{\Norm}{\lVert}{\rVert}			% Norm
\DeclarePairedDelimiter{\sprod}{\langle}{\rangle}		% spitze Klammern
\DeclarePairedDelimiter{\enbrace}{(}{)}					% runde Klammern
\DeclarePairedDelimiter{\benbrace}{\lbrack}{\rbrack}	% eckige Klammern
\DeclarePairedDelimiter{\penbrace}{\{}{\}}				% geschweifte Klammern
\newcommand{\Underbrace}[2]{{\underbrace{#1}_{#2}}} 	% Underbrace als Befehl in LaTeX-Syntax (und ohne Spacing-Probleme mit nachfolgenden Operatoren...)

\DeclareMathOperator{\prT}{(T)}		% Eigenschaft (T)
\DeclareMathOperator{\prFH}{(FH)}	% Eigenschaft (FH)
\DeclareMathOperator{\prFC}{(FC)}	% Eigenschaft (FC)
\DeclareMathOperator{\prFA}{(FA)}	% Eigenschaft (FA)
\newcommand{\subfi}{\sqsubseteq}	% Untergruppe mit endlichem Index
\newcommand{\gind}[1]{\benbrace*{#1}}	% Gruppenindex
\DeclareMathOperator{\cat}{CAT(0)}
\DeclareMathOperator{\dom}{dom}
\DeclareMathOperator{\Fix}{Fix}
\DeclareMathOperator{\Eig}{Eig}
\DeclareMathOperator{\rk}{rk}
\DeclareMathOperator{\St}{St}
\DeclareMathOperator{\Inn}{Inn}
\DeclareMathOperator{\Ind}{Ind}