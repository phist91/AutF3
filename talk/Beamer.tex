%!TEX TS-program = xelatex
% -- Author: Phil Steinhorst, p.st@wwu.de
\newcommand{\verfasser}{Phil Steinhorst}
\newcommand{\titel}{Kubische Komplexe, Kazhdans Eigenschaft $\prT$\\und Eigenschaft $\FSC$}								
\newcommand{\shorttitel}{Kubische Komplexe, Eigenschaft $\prT$, Eigenschaft $\FSC$}
\newcommand{\subtitel}{Seminar zur Gruppentheorie und Geometrie:\\ Kazhdan- und Haagerup-Eigenschaften von Gruppen}
\newcommand{\datum}{07. Mai 2015}
\newcommand{\authormail}{p.st@wwu.de}
\documentclass[10pt,aspectratio=169]{beamer}
%!TEX root = Beamer.tex
% -- Author: Phil Steinhorst, p.st@wwu.de
\usetheme{Boadilla}
\setbeamertemplate{navigation symbols}{}
\setbeamertemplate{itemize items}[square]

\title{\shorttitel}
\author{\verfasser}

\defbeamertemplate*{title page}{customized}[1][]
{
	\begin{center}
	\begin{minipage}{0.4\textwidth}
	\begin{flushleft}
	\hspace*{1em}\includegraphics[height=1cm,keepaspectratio]{../img/wwulogo.pdf}
	\end{flushleft}
	\end{minipage}
	\hfill
	\begin{minipage}{0.4\textwidth}
	\begin{flushright}
	\vspace*{0.3cm}
	\includegraphics[height=0.8cm,keepaspectratio]{../img/fb10logo.pdf} \hspace*{1em}
	\end{flushright}
	\end{minipage}
  
	\vfill
   
	\textcolor{structure}{{\LARGE \titel}}\par
	\vspace*{0.3cm}
	\subtitel\par
	\vspace*{1cm}
	\verfasser \par
	{\tt {\footnotesize \authormail}}\par
	\vspace*{0.3cm}
	{\footnotesize \datum}\par
  \end{center}
}

\usepackage{tikz}
\usepackage{tikz-cd}
\usetikzlibrary{external}
\tikzset{>=latex}
\usetikzlibrary{shapes,arrows,intersections}
\usetikzlibrary{calc,3d}
\usetikzlibrary{decorations.pathreplacing,decorations.markings}
\usetikzlibrary{angles}
\tikzset{
  every picture/.append style={
    execute at begin picture={\shorthandoff{"}},
    execute at end picture={\shorthandon{"}}
  }
}
\usetikzlibrary{quotes}
\usetikzlibrary{patterns}
\tikzset{
	schraffiert/.style={pattern=north west lines,pattern color=#1},
	schraffiert/.default=black
}
\usepackage{pgfplots}
\usepackage{mathtools} 					
\mathtoolsset{showonlyrefs, centercolon}
\usepackage{wasysym}
\usepackage{amssymb} 		
\usepackage{latexsym} 		
\usepackage{stmaryrd} 		
\usepackage{nicefrac} 		
\usepackage{cancel} 		
\usepackage{extarrows}		
\usepackage{mathdots}
\DeclareSymbolFont{bbold}{U}{bbold}{m}{n}
\DeclareSymbolFontAlphabet{\mathbbold}{bbold}
\newcommand{\mathds}[1]{\mathbb{#1}} 	
\def\mathul#1#2{\color{#1}\underline{{\color{black}#2}}\color{black}}
\usepackage{mathspec} 				
\usepackage{polyglossia} 			
\setmainlanguage[spelling=new,babelshorthands=false]{german}
\newcommand\glqq{"}
\newcommand\grqq{"}
\setallmonofonts[Scale=MatchLowercase, ItalicFont={*}]{Consolas} 
\usepackage{xltxtra}
\usepackage{fontawesome}
\usepackage[german=quotes]{csquotes}
\usepackage{booktabs}
\usepackage{wrapfig}
\usepackage{float}
\usepackage{stackrel}
\usepackage{ifthen}
\usepackage{multicol}
\usepackage[normalem]{ulem}
\setlength{\ULdepth}{1.8pt}
\newcommand{\RM}[1]{\MakeUppercase{\romannumeral #1{}}} 
\usepackage{ellipsis}									
%!TEX root = AutF3.tex
% -- Author: Phil Steinhorst, p.st@wwu.de

\newcommand{\CC}{\mathbb{C}}
\newcommand{\HH}{\mathcal{H}}
\newcommand{\MM}{\mathcal{M}}
\newcommand{\NN}{\mathbb{N}}
\newcommand{\RR}{\mathbb{R}}
\newcommand{\UU}{\mathcal{U}}
\newcommand{\ZZ}{\mathbb{Z}}

\newcommand{\bewrueck}{\glqq$\Leftarrow$\grqq:} 	% Beweis R�ckrichtung
\newcommand{\bewhin}{\glqq$\Rightarrow$\grqq:}		% Beweis Hinrichtung
\newcommand{\setone}{\{1\}}							% Einsmenge
\newcommand{\NT}{\trianglelefteq}					% Normalteiler
\newcommand{\setzero}{\{0\}}						% Nullmenge

\DeclareMathOperator{\Aut}{Aut} 					% Automorphismen
\DeclareMathOperator{\GL}{GL}						% allgemeine lineare Gruppe
\DeclareMathOperator{\id}{id} 						% Identit�t
\DeclareMathOperator{\im}{im} 						% image
\DeclareMathOperator{\Isom}{Isom}					% Isometrien
\DeclareMathOperator{\Sym}{Sym} 					% Symmetrische Gruppe

\DeclarePairedDelimiter{\abs}{\lvert}{\rvert}			% Betrag
\DeclarePairedDelimiter{\ceil}{\lceil}{\rceil}			% aufrunden
\DeclarePairedDelimiter{\floor}{\lfloor}{\rfloor}		% aufrunden
\DeclarePairedDelimiter{\Norm}{\lVert}{\rVert}			% Norm
\DeclarePairedDelimiter{\sprod}{\langle}{\rangle}		% spitze Klammern
\DeclarePairedDelimiter{\enbrace}{(}{)}					% runde Klammern
\DeclarePairedDelimiter{\benbrace}{\lbrack}{\rbrack}	% eckige Klammern
\DeclarePairedDelimiter{\penbrace}{\{}{\}}				% geschweifte Klammern
\newcommand{\Underbrace}[2]{{\underbrace{#1}_{#2}}} 	% Underbrace als Befehl in LaTeX-Syntax (und ohne Spacing-Probleme mit nachfolgenden Operatoren...)

\DeclareMathOperator{\prT}{(T)}		% Eigenschaft (T)
\DeclareMathOperator{\prFH}{(FH)}	% Eigenschaft (FH)
\DeclareMathOperator{\prFC}{(FC)}	% Eigenschaft (FC)
\DeclareMathOperator{\prFA}{(FA)}	% Eigenschaft (FA)
\newcommand{\subfi}{\sqsubseteq}	% Untergruppe mit endlichem Index
\newcommand{\gind}[1]{\benbrace*{#1}}	% Gruppenindex
\DeclareMathOperator{\cat}{CAT(0)}
\DeclareMathOperator{\dom}{dom}
\DeclareMathOperator{\Fix}{Fix}
\DeclareMathOperator{\Eig}{Eig}
\DeclareMathOperator{\rk}{rk}
\DeclareMathOperator{\St}{St}
\DeclareMathOperator{\Inn}{Inn}
\DeclareMathOperator{\Ind}{Ind}	
\date{07. Mai 2015}
\begin{document}
\begin{frame}[t]
	\titlepage
\end{frame}

\begin{frame}{Inhalt}
	\begin{itemize}
		\item Kubische Komplexe: Definitionen, Eigenschaften und Beispiele \pause
		\item Eigenschaft $\FSC$ \pause
		\item Ein Resultat über Gruppenwirkungen auf bestimmte kubische Komplexe: \pause
	\end{itemize}
	\begin{block}{Theorem}
		Sei $X$ ein vollständiger $\cat$ kubischer Komplex und $G$ eine endlich erzeugte Gruppe, die die Kazhdan-Eigenschaft $\prT$ erfüllt. Dann hat jede simpliziale Wirkung von $G$ auf $X$ einen globalen Fixpunkt.
	\end{block}
\end{frame}

\begin{frame}[c]{}
\begin{center}
	{\Huge \selectfont \textbf{Kubische Komplexe}}
\end{center}
\end{frame}

\begin{frame}{Kubische Komplexe}
	\begin{block}{Würfel, Seite, Dimension}
		\begin{itemize}
			\item $n$-\textbf{Würfel:} $C = \begin{cases}
				[0,1]^n \subseteq \RR^n, & n \geq 1 \\
				\{0\} & n=0
			\end{cases}$ \pause
			\item \textbf{Seite:} $F = F_1 \times F_2 \times \dots \times F_n \subseteq C$ 
				mit $F_i \in \{ \{0\}, \{1\}, [0,1]\}$. \pause
			\item $d_C$: euklidische Metrik des $\RR^n$ auf $C$ \pause
			\item \textbf{Dimension:} $\dim(C) := n$.
		\end{itemize}
	\end{block}	
\end{frame}

\begin{frame}[c]{Kubische Komplexe}
	\centering
	\begin{tikzpicture}[scale=2.5]
		\draw [->,very thick] (0,0) -- (0,1.4) node[above] {$x_3$};
		\draw [->,very thick] (0,0) -- (1.4,0) node[right] {$x_1$};
		\draw [->,very thick] (0,0) -- (-.7,-.7) node[below] {$x_2$};
	
		\draw (-.5,-.5) -- (-.5,.5) -- (0,1) -- (0,0) -- (-.5,-.5) -- (.5,-.5) -- (1,0) -- (0,0);
		\draw (.5,-.5) -- (.5,.5) -- (-.5,.5);
		\draw (0,1) -- (1,1) -- (.5,.5);
		\draw (1,1) -- (1,0);
		\draw<1> (1.1,1.1) node[right]{$[0,1]^3$}; 
		
		\onslide<2->{
			\draw[schraffiert=orange] (-.5,-.5) -- (-.5,.5) -- (0,1) -- (0,0) -- (-.5,-.5);
			\draw [color=orange] (-.6,0) node[left] {$\{0\} \times [0,1]^2$};
		}
		
		\onslide<3->{
			\draw[schraffiert=red] (.5,-.5) -- (.5, .5) -- (1,1) -- (1,0) -- (.5,-.5);
			\draw [color=red] (.7,-.4) node[right] {$\{1\} \times [0,1]^2$};
		}
		
		\onslide<4->{
			\draw[schraffiert=blue] (-.5,.5) -- (.5,.5) -- (1,1) -- (0,1) -- (-.5,.5);
			\draw [color=blue] (.1,1.2) node[right] {$[0,1]^2 \times \{1\}$};
		}
		
		\onslide<5->{
			\draw[ultra thick, color=teal] (.5,.5) -- (1,1);
			\draw [color=teal] (1.05,.9) node[right] {$\{1\} \times [0,1] \times \{1\}$};
		}
	\end{tikzpicture}
	\begin{exampleblock}<6->{Bemerkung}
		Der Schnitt zweier Seiten von $C$ ist entweder leer oder wieder eine Seite von $C$.
	\end{exampleblock}
\end{frame}

\begin{frame}[t]{Kubische Komplexe}
	\begin{block}{Klebung}
		$C_1, C_2$: Würfel, $F_1 \subseteq C_1, F_2 \subseteq C_2$: Seiten \pause \\
		Eine \textbf{Klebung} ist eine bijektive Isometrie $\varphi\colon F_1 \rightarrow F_2$. \pause
	\end{block} \vspace{.5cm}
	\centering
	\begin{tikzpicture}[scale=2.5]
		\draw (1,0) -- (0,0) -- (0,1) -- (1,1);
		\draw [ultra thick] (1,0) -- (1,1);
		\draw (0.2,-.15) node {$C_1$};
		\draw (1,-.15) node {$F_1$};
		
		\draw (3,-.5) -- (4,-.5) -- (4,.5) -- (3,.5) -- (3.5,1) -- (4.5,1) -- (4.5,0) -- (4,-.5);
		\draw (4,.5) -- (4.5,1);
		\draw [ultra thick] (3,.5) -- (3,-.5);
		\draw (3,-.65) node {$F_2$};
		\draw (3.8,-.65) node {$C_2$};
		
		\onslide<4->{		
		\draw (2.95,.16) -- (3.05,.16) node [pos=2.8] {$\varphi(x)$};
		\draw (2.95,-.16) -- (3.05,-.16) node [pos=2.8] {$\varphi(y)$};	
		
		\draw (.95,0.33) -- (1.05,0.33) node[pos=-.8] {$y$};
		\draw (.95,0.66) -- (1.05,0.66) node[pos=-.8] {$x$};
		
		\draw [->, rounded corners=12pt] (1.1,.66) -- (1.6,.66) -- (2.4,.16) -- (2.9,.16);
		\draw [->, rounded corners=12pt] (1.1,.33) -- (1.6,.33) -- (2.4,-.16) -- (2.9,-.16); }
		
		\draw<5-> (2,-.7) node[align=center]{$d_{C_1}(x,y) = d_{C_2}(\varphi(x),\varphi(y))$};
	\end{tikzpicture}
\end{frame}

\begin{frame}[c]{Kubische Komplexe}
	\begin{block}{Definition: Kubischer Komplex}
		$\mathcal{C}$: Familie von Würfeln, $\mathcal{S}$: Familie von Klebungen von Würfeln in $\mathcal{C}$  \pause
		mit folgenden Eigenschaften:
		\begin{enumerate}[(i)]
			\item Kein Würfel ist mit sich selbst verklebt.
			\item Je zwei Würfel aus $\mathcal{C}$ sind höchstens einmal miteinander verklebt. \pause
		\end{enumerate}
		Äquivalenzrelation auf $\sqcup_{C \in \mathcal{C}} C$: \pause
		\[ x \sim y :\Leftrightarrow \text{ es existiert ein } \varphi \in \mathcal{S} \text{ mit } x \in \dom(\varphi) \text{ und } \varphi(x) = y \pause \]
	\end{block}
	\centering
	\begin{tikzpicture}[scale=2]
		\draw (1,0) -- (0,0) -- (0,1) -- (1,1);
		\draw [ultra thick] (1,0) -- (1,1);
		
		\draw (3,-.5) -- (4,-.5) -- (4,.5) -- (3,.5) -- (3.5,1) -- (4.5,1) -- (4.5,0) -- (4,-.5);
		\draw (4,.5) -- (4.5,1);
		\draw [ultra thick] (3,.5) -- (3,-.5);
		
		\draw (2.95,.16) -- (3.05,.16) node [pos=2.8] {$\varphi(x)$};
		\draw (2.95,-.16) -- (3.05,-.16) node [pos=2.8] {$\varphi(y)$};	
		
		\draw (.95,0.33) -- (1.05,0.33) node[pos=-.8] {$y$};
		\draw (.95,0.66) -- (1.05,0.66) node[pos=-.8] {$x$};
		
		\draw [->, rounded corners=12pt] (1.1,.66) -- (1.6,.66) -- (2.4,.16) -- (2.9,.16);
		\draw [->, rounded corners=12pt] (1.1,.33) -- (1.6,.33) -- (2.4,-.16) -- (2.9,-.16);
	\end{tikzpicture}
\end{frame}

\begin{frame}[c]{Kubische Komplexe}
	\begin{block}{Definition: Kubischer Komplex (Forts.)}
		\textbf{Kubischer Komplex:} 
		\[ X := \enbrace*{\bigsqcup_{C \in \mathcal{C}} C} \diagup \sim. \]
	\end{block}
	\centering
	\begin{tikzpicture}[scale=2.5]
		\draw (2,1) -- (0,1) -- (0,0) -- (1,0) -- (1,1) -- (1.5,1.5) -- (2.5,1.5) -- (2,1) -- (2,0) -- (1,0);
		\draw (2,0) -- (2.5,.5) -- (2.5,1.5);
		\draw (1,-.15) node {$X$};
		\draw (.95,.33) -- (1.05,.33) node[pos=2.7] {$[y]_\sim$};
		\draw (.95,.66) -- (1.05,.66) node[pos=2.7] {$[x]_\sim$};
	\end{tikzpicture}
\end{frame}

\begin{frame}[c]{Beispiele für Kubische Komplexe}
	\centering
	\begin{tikzpicture}[scale=1.5]
		\draw (0,0) -- (.5,.5) -- (1,0) -- (2,0) -- (2.5,-.5);
		\draw (2,0) -- (2.5,.5);
		\draw (0,1) -- (.5,.5) -- (1,1) -- (2,1);
		\draw (1,1) -- (1,2);
		\draw (0,0) node[fill,circle,inner sep=1pt]{};
		\draw (.5,.5) node[fill,circle,inner sep=1pt]{};
		\draw (0,1) node[fill,circle,inner sep=1pt]{};
		\draw (1,1) node[fill,circle,inner sep=1pt]{};
		\draw (1,2) node[fill,circle,inner sep=1pt]{};
		\draw (2,1) node[fill,circle,inner sep=1pt]{};
		\draw (1,0) node[fill,circle,inner sep=1pt]{};
		\draw (2,0) node[fill,circle,inner sep=1pt]{};
		\draw (2.5,-.5) node[fill,circle,inner sep=1pt]{};
		\draw (2.5,.5) node[fill,circle,inner sep=1pt]{};
	\end{tikzpicture} \pause \hspace{1cm}
	\begin{tikzpicture}[scale=1.3]
		\draw[->,very thick] (-.2,0) -- (2.3,0);
		\draw (-.2,1) -- (2.3,1); 
		\draw (-.2,2) -- (2.3,2); 
		\draw (.3,2.5) -- (2.8,2.5);
		
		\draw[->,very thick] (0,-.2) -- (0,2.3);
		\draw (1,-.2) -- (1,2.3);
		\draw (2,-.2) -- (2,2.3);
		\draw (2.5,.5) -- (2.5,2.8);
		
		\draw (-.1,1.9) -- (.5,2.5);
		\draw (.9,1.9) -- (1.5,2.5);
		\draw (1.9,1.9) -- (2.5,2.5);
		\draw (1.9,.9) -- (2.5,1.5);
		\draw (1.9,-.1) -- (2.5,.5);
		
		\draw (.5,2.5) -- (.5,2.8);
		\draw (1.5,2.5) -- (1.5,2.8);
		\draw (2.5,.5) -- (2.8,.5);
		\draw (2.5,1.5) -- (2.8,1.5);
		
		\draw (0,0) -- (-.1,-.1);
		\draw (1,0) -- (.9,-.1);
		\draw (0,1) -- (-.1,.9);
		\draw (1,1) -- (.9,.9);
	\end{tikzpicture} \pause \hspace{1cm}
	\begin{tikzpicture}[scale=1.3]
		\draw (0,1) -- (3,1);
		\draw (0,2) -- (3,2);
		\draw (1,0) -- (1,3);
		\draw (2,0) -- (2,3);
		
		\draw[color=green,very thick] (0,3) -- (3,3);
		\draw[->,color=green,very thick] (0,3) -- (1.7,3);
		\draw[->,color=green,very thick] (0,3) -- (1.53,3);
		\draw[color=green,very thick] (0,0) -- (3,0);
		\draw[->,color=green,very thick] (0,0) -- (1.7,0);
		\draw[->,color=green,very thick] (0,0) -- (1.53,0);
		
		\draw[color=red,very thick] (0,3) -- (0,0);
		\draw[->,color=red,very thick] (0,3) -- (0,1.47);
		\draw[color=red,very thick] (3,3) -- (3,0);
		\draw[->,color=red,very thick] (3,3) -- (3,1.47);
	\end{tikzpicture}
\end{frame}

\begin{frame}[c]{Zusammenhang}
	\begin{block}{Weg}
		$X$: kubischer Komplex, $x,y \in X$. \\
		Ein \textbf{Weg} von $x$ nach $y$ in $X$ ist eine endliche Folge $\sigma = (x_0, \dots, x_m) \subseteq X$ mit $x_0 = x$ und $x_m = y$,\\ sodass gilt: \\ \pause Für alle $0 \leq i \leq m-1$ existiert ein Würfel $C_i$ von $X$ mit $x_i, x_{i+1} \in C_i$. \pause
	\end{block}
	\hspace*{1.5cm}
	\begin{tikzpicture}[scale=1.8]
		\draw (1,0) -- (1,1) -- (0,1) -- (0,0) -- (1,0) -- (1.5,.5) -- (1.5,1.5) -- (.5,1.5) -- (0,1);
		\draw (1,1) -- (1.5,1.5);
		
		\draw (0,0) -- (-.5,-.5) -- (.5,-.5) -- (1,0);
		\draw (1.5,.5) -- (2.5,.5) -- (2.5,1.5) -- (1.5,1.5);
		\draw (2.5,1.5) -- (3.5,1.5) -- (3.5,.5) -- (2.5,.5);
		\draw (3.5,.5) -- (3,0) -- (2,0) -- (2.5,.5);
		\draw (2,0) -- (2,-1) -- (3,-1) -- (3,0);
		\draw (3,-1) -- (3.5,-.5) -- (3.5,.5);
		
		\onslide<3-4>{
			\draw [dashed, color=red] (-.3,-.4) node[fill,circle,inner sep=1.5pt]{} 
			-- (.8,0) node[fill,circle,inner sep=1.5pt]{} 
			-- (1.5,.8) node[fill,circle,inner sep=1.5pt]{} 
			-- (2.5,.8) node[fill,circle,inner sep=1.5pt]{} 
			-- (3,.5) node[fill,circle,inner sep=1.5pt]{} 
			-- (2.8,-.8) node[fill,circle,inner sep=1.5pt]{};
			
			\draw[color=red] (-.3,-.4) node[anchor = south east]{$x_0$};
			\draw[color=red] (.8,0) node[anchor = south east]{$x_1$};
			\draw[color=red] (1.5,.8) node[anchor = south east]{$x_2$};
			\draw[color=red] (2.5,.8) node[anchor = south east]{$x_3$};
			\draw[color=red] (3,.5) node[anchor = south west]{$x_4$};
			\draw[color=red] (2.8,-.8) node[anchor = east]{$x_5$};
		}
		
		\onslide<3->{
			\draw (0,-.5) node[below]{$A$};
			\draw (0,1) node[left]{$B$};
			\draw (2,1.5) node[below]{$C$};
			\draw (3.5,1) node[right]{$D$};
			\draw (3.5,-.5) node[right]{$E$};
		}
		
		\draw<4> (5,1.3) node[below,align=left]{$x_0,x_1 \in A$ \\ $x_1,x_2 \in B$ \\ $x_2,x_3 \in C$ \\ $x_3,x_4 \in D$ \\ $x_4,x_5 \in E$ \\ $\rightarrow$ Weg von $x_0$ nach $x_5$};
		
		\onslide<5-6>{
			\draw [dashed, color=red] (-.3,-.4) node[fill,circle,inner sep=1.5pt]{} 
			-- (.8,0) node[fill,circle,inner sep=1.5pt]{} 
			-- (1.5,.8) node[fill,circle,inner sep=1.5pt]{} 
			-- (3,.5) node[fill,circle,inner sep=1.5pt]{} 
			-- (2.8,-.8) node[fill,circle,inner sep=1.5pt]{};
			
			\draw[color=red] (-.3,-.4) node[anchor = south east]{$x_0$};
			\draw[color=red] (.8,0) node[anchor = south east]{$x_1$};
			\draw[color=red] (1.5,.8) node[anchor = south east]{$x_2$};
			\draw[color=red] (3,.5) node[anchor = south west]{$x_3$};
			\draw[color=red] (2.8,-.8) node[anchor = east]{$x_4$};
		}
		
		\draw<6> (5,1.3) node[below,align=left]{$x_2 \in B,C$ \\ $x_3 \in D, E$ \\ $\rightarrow$ kein Weg!};
		
		\onslide<7-8>{
		\draw [dashed, color=red] (-.3,-.4) node[fill,circle,inner sep=1.5pt]{} 
			-- (.8,0) node[fill,circle,inner sep=1.5pt]{} 
			-- (1.5,.6) node[fill,circle,inner sep=1.5pt]{} 
			-- (2.5,.6) node[fill,circle,inner sep=1.5pt]{} 
			-- (3.2,1.2) node[fill,circle,inner sep=1.5pt]{} 
			-- (3,.5) node[fill,circle,inner sep=1.5pt]{} 
			-- (2.8,-.8) node[fill,circle,inner sep=1.5pt]{};
			
			\draw[color=red] (-.3,-.4) node[anchor = south east]{$x_0$};
			\draw[color=red] (.8,0) node[anchor = south east]{$x_1$};
			\draw[color=red] (1.5,.6) node[anchor = south east]{$x_2$};
			\draw[color=red] (2.5,.6) node[anchor = south east]{$x_3$};
			\draw[color=red] (3.2,1.2) node[anchor = south east]{$x_4$};
			\draw[color=red] (3,.5) node[anchor = west]{$x_5$};
			\draw[color=red] (2.8,-.8) node[anchor = east]{$x_6$};
		}
		
		\draw<8> (5,1.3) node[below,align=left]{$x_0,x_1 \in A$ \\ $x_1,x_2 \in B$ \\ $x_2,x_3 \in C$ \\ $x_3,x_4 \in D$ \\ $x_4,x_5 \in D$ \\ $x_5,x_6 \in E$ \\ $\rightarrow$ Weg von $x_0$ nach $x_6$};
	\end{tikzpicture}
\end{frame}

\begin{frame}[c]{Zusammenhang}
	\centering
	\begin{tikzpicture}[scale=1.5]
		\draw (1,0) -- (1,1) -- (0,1) -- (0,0) -- (1,0) -- (1.5,.5) -- (1.5,1.5) -- (.5,1.5) -- (0,1);
		\draw (1,1) -- (1.5,1.5);
		
		\draw (0,0) -- (-.5,-.5) -- (.5,-.5) -- (1,0);
		\draw (1.5,.5) -- (2.5,.5) -- (2.5,1.5) -- (1.5,1.5);
		\draw (2.5,1.5) -- (3.5,1.5) -- (3.5,.5) -- (2.5,.5);
		\draw (3.5,.5) -- (3,0) -- (2,0) -- (2.5,.5);
		\draw (2,0) -- (2,-1) -- (3,-1) -- (3,0);
		\draw (3,-1) -- (3.5,-.5) -- (3.5,.5);
		
		\draw [dashed, color=red] (-.3,-.4) node[fill,circle,inner sep=1.5pt]{} 
		-- (.8,0) node[fill,circle,inner sep=1.5pt]{} 
		-- (1.5,.6) node[fill,circle,inner sep=1.5pt]{} 
		-- (2.5,.6) node[fill,circle,inner sep=1.5pt]{} 
		-- (3.2,1.2) node[fill,circle,inner sep=1.5pt]{} 
		-- (3,.5) node[fill,circle,inner sep=1.5pt]{} 
		-- (2.8,-.8) node[fill,circle,inner sep=1.5pt]{};
		
		\draw[color=red] (-.3,-.4) node[anchor = south east]{$x_0$};
		\draw[color=red] (.8,0) node[anchor = south east]{$x_1$};
		\draw[color=red] (1.5,.6) node[anchor = south east]{$x_2$};
		\draw[color=red] (2.5,.6) node[anchor = south east]{$x_3$};
		\draw[color=red] (3.2,1.2) node[anchor = south east]{$x_4$};
		\draw[color=red] (3,.5) node[anchor = west]{$x_5$};
		\draw[color=red] (2.8,-.8) node[anchor = east]{$x_6$};
	\end{tikzpicture}
	\begin{block}{Länge eines Weges}
		$X$: kubischer Komplex, $\sigma = (x_0, \dots x_m)$: Weg in $X$ mit $x_i, x_{i+1} \in C_i$ \pause \\
		Die \textbf{Länge} von $\sigma$ ist gegeben durch:
		\[ \ell(\sigma) := \sum\limits_{i=0}^{m-1} d_{C_i} (x_i,x_{i+1})\]
	\end{block}
\end{frame}

\begin{frame}[c]{Kubische Komplexe als metrische Räume}
	\begin{block}{Länge eines Weges}
		$X$: kubischer Komplex, $\sigma = (x_0, \dots x_m)$: Weg in $X$ mit $x_i, x_{i+1} \in C_i$ \\
		Die \textbf{Länge} von $\sigma$ ist gegeben durch:
		\[ \ell(\sigma) := \sum\limits_{i=0}^{m-1} d_{C_i} (x_i,x_{i+1})\]
	\end{block}
	\begin{block}{Längenmetrik}
		$X$: \textbf{wegzusammenhängender} kubischer Komplex, d.h. für alle $x,y \in X$ existiert Weg von $x$ nach $y$. \pause \\
		Definiere die \textbf{Längenmetrik} durch:
		\begin{equation}
		\begin{aligned}
			d\colon X \times X &\longrightarrow \RR \\
			(x,y) &\longmapsto \inf \{\ell(\sigma) : \sigma \text{ ist Weg von } x \text{ nach } y \text{ in } X\}
		\end{aligned}
		\end{equation}
	\end{block}
\end{frame}

\begin{frame}[c]{Kubische Komplexe als metrische Räume}
	\centering
	\begin{tikzpicture}[scale=1.5]
		\draw[fill,color=lightgray] (0,0) -- (1,0) -- (2,0) -- (2.5,0.5) -- (3.5,0.5) -- (3.5,1.5) -- (2.5,1.5) -- (1.5,1.5) -- (1,1) -- (0,1) -- (0,0);
		\draw (0,0) -- (1,0) -- (2,0) -- (2.5,0.5) -- (3.5,0.5) -- (3.5,1.5) -- (2.5,1.5) -- (1.5,1.5) -- (1,1) -- (0,1) -- (0,0);
		\draw (1,0) -- (1,1) -- (2,1) -- (2.5,1.5) -- (2.5,0.5);
		\draw (2,0) -- (2,1);
		\draw (1.5,0) node[below]{$X$};
	\end{tikzpicture} \hspace{2cm}
	\begin{tikzpicture}[scale=1.5]
		\onslide<2->{
			\draw (0,0) node[fill,circle,inner sep=1pt]{} -- (1,0) node[fill,circle,inner sep=1pt]{} -- (2,0) node[fill,circle,inner sep=1pt]{} -- (2.5,0.5) node[fill,circle,inner sep=1pt]{} -- (3.5,0.5) node[fill,circle,inner sep=1pt]{} -- (3.5,1.5) node[fill,circle,inner sep=1pt]{} -- (2.5,1.5) node[fill,circle,inner sep=1pt]{} -- (1.5,1.5) node[fill,circle,inner sep=1pt]{} -- (1,1) node[fill,circle,inner sep=1pt]{} -- (0,1) node[fill,circle,inner sep=1pt]{} -- (0,0) node[fill,circle,inner sep=1pt]{};
			\draw (1,0) -- (1,1) -- (2,1) node[fill,circle,inner sep=1pt]{} -- (2.5,1.5) -- (2.5,0.5) -- (1.5,0.5) node[fill,circle,inner sep=1pt]{} -- (1,0);
			\draw (2,0) -- (2,1) ;
			\draw (1.5,0.5) -- (1.5,1.5);
			\draw (1.5,0) node[below]{$X^{(1)}$};
		}
		
		\onslide<4->{
			\draw[color=red] (1,0) node[fill,circle,inner sep=1pt]{} -- (1,1) node[fill,circle,inner sep=1pt]{} -- (2,1) node[fill,circle,inner sep=1pt]{} -- (2.5,1.5) node[fill,circle,inner sep=1pt]{} -- (3.5,1.5) node[fill,circle,inner sep=1pt]{};
			\draw (1,-.1) node[anchor=north east] {$x$};
			\draw (3.6,1.5) node[right] {$y$};
		}
	\end{tikzpicture}
	\begin{block}{simpliziale Metrik}<3->
		Die Abbildung
		\begin{equation}
		\begin{aligned}
			D\colon X^{(0)} \times X^{(0)} &\longrightarrow \RR \\
			(x,y) &\longmapsto \inf \{\ell(\sigma) : \sigma \text{ ist Weg von } x \text{ nach } y \textcolor{red}{\text{ in } X^{(1)}}\}
		\end{aligned}
		\end{equation}
		heißt \textbf{simpliziale Metrik} auf $X$. Es gilt $d(x,y) \leq D(x,y)$.
	\end{block}
\end{frame}

\begin{frame}{Hyperebenen in kubischen Komplexen}
	\begin{block}{Mittelwürfel}
		Sei $C := [0,1]^n$. Der $i$-te \textbf{Mittelwürfel}, $1 \leq i \leq n$, ist gegeben durch
		\[
			M_i := \penbrace*{x \in C : x_i = \frac{1}{2}}= [0,1]^{i-1} \times \penbrace*{\frac{1}{2}} \times [0,1]^{n-i}
		\]
	\end{block}
	\centering
	\begin{tikzpicture}[scale=2]
		\onslide<2->{
			\draw [->,very thick] (0,0) -- (0,1.4) node[above] {$x_3$};
			\draw [->,very thick] (0,0) -- (1.4,0) node[right] {$x_1$};
			\draw [->,very thick] (0,0) -- (-.7,-.7) node[below] {$x_2$};
			
			\draw (-.5,-.5) -- (-.5,.5) -- (0,1) -- (0,0) -- (-.5,-.5);
			\draw (.5,-.5) -- (.5, .5) -- (1,1) -- (1,0) -- (.5,-.5);
			\draw (-.5,.5) -- (.5,.5) -- (1,1) -- (0,1) -- (-.5,.5);
			\draw (-.5,-.5) -- (.5,-.5);
		}
		
		\onslide<3,6>{		
			\draw [color=teal,schraffiert=teal] (0,-.5) -- (.5,0) -- (.5,1) -- (0,.5) -- (0,-.5);
			\draw [color=teal] (0,-.5) node[below]{$M_1$};
		}
		
		\onslide<4,6>{
			\draw [color=red,schraffiert=red] (-.25,-.25) -- (.75,-.25) -- (.75,.75) -- (-.25,.75) -- (-.25,-.25);
					\draw [color=red] (-.4,.85) node{$M_2$};
		}
		
		\onslide<5,6>{
			\draw [color=blue,schraffiert=blue] (-.5,0) -- (.5,0) -- (1,.5) -- (0,.5) -- (-.5,0);
			\draw [color=blue] (1.0,.5) node[right]{$M_3$};
		}
	\end{tikzpicture}
\end{frame}

\begin{frame}{Hyperebenen in kubischen Komplexen}
	\begin{columns}[c]
		\begin{column}{.4\textwidth}
			\only<1-6>{
			\begin{block}{quadratäquivalent}
				$X$: kubischer Komplex. Betrachte die durch
				\[
					\begin{array}{rcl}
						e \parallel e' & :\Leftrightarrow & e \text{ liegt gegenüber } e' \text{ in} \\
						& & \text{einem } 2\text{-Würfel von } X
					\end{array}
				\]
				erzeugte Äquivalenzrelation auf der Menge der Kanten ($1$-Würfel) von $X$. $e$ und $e'$ heißen \textbf{quadratäquivalent}, wenn gilt: $e \parallel e'$.
			\end{block}
			}
			\only<7->{
			\begin{block}{transversal}
				Ein Mittelwürfel $M$ heißt \textbf{transversal} zu $[e]_{\parallel}$ (schreibe: $M \pitchfork [e]_{\parallel}$), wenn $M \cap X^{(1)}$ aus Mittelpunkten von Kanten in $[e]_{\parallel}$ besteht.
			\end{block}
			\begin{block}<12->{Hyperebene}
				Die Vereinigung aller zu $[e]_\parallel$ transversalen Mittelwürfel heißt \textbf{Hyperebene} zu $e$.
				\[ H(e) := \bigcup_{M \pitchfork [e]_\parallel} M\]
			\end{block}
			}
		\end{column}
		\begin{column}{.58\textwidth}
			\centering
			\begin{tikzpicture}[scale=2]
				\draw<2-> (2,2) -- (1,2) -- (1,1) -- (2,1) -- (2,0) -- (1,0) -- (1,1) -- (0,1) -- (0,2) -- (1,2) -- (1.5,2.5) -- (3.5,2.5) -- (3.5,1.5) -- (2.5,1.5) -- (2,1) -- (2,2) -- (2.5,2.5) -- (2.5,1.5);
				\draw<3-> [color=red,very thick] (2,2) -- (2,1);
				\draw<3-4> [color=red] (2,1) node[anchor=north west] {$e$};
				\onslide<4->{
					\draw [color=red,very thick] (1,1) -- (1,2);
					\draw [color=red,very thick] (2.5,2.5) -- (2.5,1.5);
				}
				\onslide<5->{
					\draw [color=red,very thick] (0,1) -- (0,2);
					\draw [color=red,very thick] (3.5,2.5) -- (3.5,1.5);
					\draw [dashed] (1,1) -- (1.5,1.5);
					\draw [dashed] (1.5,1.5) -- (2.5,1.5);
					\draw [color=red,ultra thick,dashed] (1.5,2.5) -- (1.5,1.5);
					\draw [color=red] (2,1) node[anchor=north west] {$[e]_{\parallel}$};
				}
				\onslide<6>{
					\draw [color=teal,very thick] (0,2) -- (1,2);
					\draw [color=teal,very thick] (1,0) -- (1,1);
					\draw [color=teal] (.5,2)  node[above] {$f_1$};
					\draw [color=teal] (1,.5)  node[left] {$f_2$};
					\draw [color=teal] (2.7,.7)  node[right,align=left] {$f_1,f_2 \not \parallel e$};
				}
				\draw<8-10> [color=blue] (2.7,.7) node[right] {$M \pitchfork [e]_\parallel$};
				
				\onslide<8>{
					\draw [color=blue,very thick] (0,1.5) -- (1,1.5);
					\draw [color=blue] (.5,1.5) node[below] {$M$};
				}

				\onslide<9>{
					\draw [color=blue,schraffiert=blue,very thick] (1,1.5) -- (2,1.5) -- (2.5,2) -- (1.5,2) -- (1,1.5);
					\draw [color=blue] (1,1.5) node[left] {$M$};
				}
				
				\onslide<10>{
					\draw [color=blue,very thick] (2.5,2) -- (3.5,2);
					\draw [color=blue] (3,2) node[below] {$M$};
				}
				
				\onslide<11>{
					\draw [color=teal,very thick,schraffiert=teal] (1.25,1.25) -- (2.25,1.25) -- (2.25,2.25) -- (1.25,2.25) -- (1.25,1.25);
					\draw [color=teal,very thick] (1,.5) -- (2,.5);
					\draw [color=teal] (1.25,2.25) node[left,align=right] {$M_1$};
					\draw [color=teal] (2,.5) node[right,align=left] {$M_2$};
					\draw [color=teal] (2.7,.7) node[right] {$M_1, M_2 \not\pitchfork [e]_\parallel$};
				}
				
				\onslide<13->{
					\draw [color=blue,very thick] (0,1.5) -- (1,1.5);
					\draw [color=blue,schraffiert=blue,very thick] (1,1.5) -- (2,1.5) -- (2.5,2) -- (1.5,2) -- (1,1.5);
					\draw [color=blue,very thick] (2.5,2) -- (3.5,2);
					\draw [color=blue] (3,2) node[below] {$H(e)$};
				}
			\end{tikzpicture}
		\end{column}
	\end{columns}
\end{frame}

\begin{frame}{Hyperebenen in kubischen Komplexen}
	\begin{exampleblock}{Bemerkung}
		Eine Hyperebene $H$ partitioniert einen vollständigen $\cat$ kubischen Komplex $X$ in die drei disjunkten Teilmengen $U^-, U^+$ und $H$. $U^-$ und $U^+$ heißen \textbf{Halbräume} von $X$.
	\end{exampleblock}
	\centering
	\begin{tabular}{ccc}
		\begin{tikzpicture}
			\draw [color=purple,schraffiert=purple] (0,1) -- (.5,1) -- (.5,2) -- (0,2) -- (0,1);
			\draw [color=teal,schraffiert=teal] (.5,1) -- (.5,2) -- (1,2) -- (1.5,2.5) -- (3.5,2.5) -- (3.5,1.5) -- (2.5,1.5) -- (2,1) -- (2,0) -- (1,0) -- (1,1) -- (.5,1);
			\draw [color=blue,ultra thick] (.5,1) -- (.5,2);
			
			\draw [thick] (2,2) -- (0,2) -- (0,1) -- (2,1) -- (2,0) -- (1,0) -- (1,2) -- (1.5,2.5) -- (3.5,2.5) -- (3.5,1.5) -- (2.5,1.5) -- (2,1) -- (2,2) -- (2.5,2.5) -- (2.5,1.5);
			 			
 			\draw [color=teal] (3.55,2) node[right] {$U^+$};
 			\draw [color=purple] (.25,1) node[below] {$U^-$};
 			\draw [color=blue] (.5,2) node[above] {$H$};
		\end{tikzpicture}
	&
		\begin{tikzpicture}
			\draw [color=purple,schraffiert=purple] (1.5,0) -- (1,0) -- (1,1) -- (0,1) -- (0,2) -- (1,2) -- (1.5,2.5) -- (2,2.5) -- (1.5,2) -- (1.5,0);
			\draw [color=teal,schraffiert=teal] (1.5,0) -- (1.5,2) -- (2,2.5) -- (3.5,2.5) -- (3.5,1.5) -- (2.5,1.5) -- (2,1) -- (2,0) -- (1.5,0);
			\draw [color=blue,ultra thick] (2,2.5) -- (1.5,2) -- (1.5,0);
			
			\draw [thick] (2,2) -- (0,2) -- (0,1) -- (2,1) -- (2,0) -- (1,0) -- (1,2) -- (1.5,2.5) -- (3.5,2.5) -- (3.5,1.5) -- (2.5,1.5) -- (2,1) -- (2,2) -- (2.5,2.5) -- (2.5,1.5);
			 			
 			\draw [color=teal] (2.1,1) node[right] {$U^+$};
 			\draw [color=purple] (.25,1) node[below] {$U^-$};
 			\draw [color=blue] (2,2.5) node[above]{$H$};
		\end{tikzpicture}
	&
		\begin{tikzpicture}
			\draw [color=purple,schraffiert=purple] (3,2.5) -- (3,1.5) -- (2.5,1.5) -- (2,1) -- (2,0) -- (1,0) -- (1,1) -- (0,1) -- (0,2) -- (1,2) -- (1.5,2.5) -- (3,2.5);
			\draw [color=teal,schraffiert=teal] (3,2.5) -- (3.5,2.5) -- (3.5,1.5) -- (3,1.5) -- (3,2.5);
			\draw [color=blue,ultra thick] (3,2.5) -- (3,1.5);
			
			\draw [thick] (2,2) -- (0,2) -- (0,1) -- (2,1) -- (2,0) -- (1,0) -- (1,2) -- (1.5,2.5) -- (3.5,2.5) -- (3.5,1.5) -- (2.5,1.5) -- (2,1) -- (2,2) -- (2.5,2.5) -- (2.5,1.5);
			 			
 			\draw [color=teal] (3.55,2) node[right] {$U^+$};
 			\draw [color=purple] (.5,1) node[below]{$U^-$};
 			\draw [color=blue] (3,1.5) node[below]{$H$};
		\end{tikzpicture}
	\\
		\begin{tikzpicture}
	 		\draw [color=teal,schraffiert = teal] (0,1.5) -- (0,2) -- (1,2) -- (1.5,2.5) -- (3.5,2.5) -- (3.5,2) -- (2.5,2) -- (2,1.5) -- (0,1.5);
 			\draw [color=purple,schraffiert = purple] (0,1) -- (1,1) -- (1,0) -- (2,0) -- (2,1) -- (2.5,1.5) -- (3.5,1.5) -- (3.5,2) -- (2.5,2) -- (2,1.5) -- (0,1.5) -- (0,1);
 			\draw [color=blue, ultra thick] (0,1.5) -- (2,1.5) -- (2.5,2) -- (3.5,2);
 			
 			\draw [thick] (2,2) -- (0,2) -- (0,1) -- (2,1) -- (2,0) -- (1,0) -- (1,2) -- (1.5,2.5) -- (3.5,2.5) -- (3.5,1.5) -- (2.5,1.5) -- (2,1) -- (2,2) -- (2.5,2.5) -- (2.5,1.5);
 			
 			\draw [color=teal] (0.5,2.05) node[above] {$U^+$};
 			\draw [color=purple] (2.05,.5) node[right] {$U^-$};
 			\draw [color=blue] (3.55,2) node[right] {$H$};
	 	\end{tikzpicture}
	 &
		\begin{tikzpicture}
 	 		\draw [color=teal,schraffiert = teal] (1,.5) -- (1,1) -- (0,1) -- (0,2) -- (1,2) -- (1.5,2.5) -- (3.5,2.5) -- (3.5,1.5) -- (2.5,1.5) -- (2,1) -- (2,.5) -- (1,.5);
  			\draw [color=purple,schraffiert = purple] (1,.5) -- (1,0) -- (2,0) -- (2,.5) -- (1,.5);
  			\draw [color=blue, ultra thick] (1,.5) -- (2,.5);
  			
  			\draw [thick] (2,2) -- (0,2) -- (0,1) -- (2,1) -- (2,0) -- (1,0) -- (1,2) -- (1.5,2.5) -- (3.5,2.5) -- (3.5,1.5) -- (2.5,1.5) -- (2,1) -- (2,2) -- (2.5,2.5) -- (2.5,1.5);
  			
  			\draw [color=teal] (0.5,2.05) node[above] {$U^+$};
  			\draw [color=purple] (2.05,.2) node[right] {$U^-$};
  			\draw [color=blue] (1,.5) node[left,align=right] {$H$};
 	 	\end{tikzpicture}
	   & 
	    \begin{tikzpicture}
   	 		\draw [color=teal,schraffiert = teal] (1.25,2.25) -- (2.25,2.25) -- (2.25,1.25) -- (2,1) -- (2,0) -- (1,0) -- (1,1) -- (0,1) -- (0,2) -- (1,2) -- (1.25,2.25);
   			\draw [color=purple,schraffiert = purple] (1.25,2.25) -- (2.25,2.25) -- (2.25,1.25) -- (2.5,1.5) -- (3.5,1.5) -- (3.5,2.5) -- (1.5,2.5) -- (1.25,2.25);
   			\draw [color=blue, ultra thick] (1.25,2.25) -- (2.25,2.25) -- (2.25,1.25);
   			
   			\draw [thick] (2,2) -- (0,2) -- (0,1) -- (2,1) -- (2,0) -- (1,0) -- (1,2) -- (1.5,2.5) -- (3.5,2.5) -- (3.5,1.5) -- (2.5,1.5) -- (2,1) -- (2,2) -- (2.5,2.5) -- (2.5,1.5);
   			
   			\draw [color=purple] (3.55,2) node[right] {$U^-$};
 			\draw [color=teal] (.5,1) node[below]{$U^+$};
 			\draw [color=blue] (2.25,1.25) node[right]{$H$};
   	 	\end{tikzpicture} \\ 
	\end{tabular} 
\end{frame}

\begin{frame}{Beispiele für Hyperebenen}
	\begin{tikzpicture}[scale=2.5]
		\onslide<1-5>{
			\draw (1,2) -- (1,1);
			\draw (2,1) -- (1,1) -- (0,0);
			\draw (0,1) -- (1,0) -- (2,0) -- (2.5,.5);
			\draw (2,0) -- (2.5,-.5);
			
			\draw (0,0) node[fill,circle,inner sep=2pt]{};
			\draw (.5,.5) node[fill,circle,inner sep=2pt]{};
			\draw (0,1) node[fill,circle,inner sep=2pt]{};
			\draw (1,1) node[fill,circle,inner sep=2pt]{};
			\draw (1,2) node[fill,circle,inner sep=2pt]{};
			\draw (2,1) node[fill,circle,inner sep=2pt]{};
			\draw (1,0) node[fill,circle,inner sep=2pt]{};
			\draw (2,0) node[fill,circle,inner sep=2pt]{};
			\draw (2.5,-.5) node[fill,circle,inner sep=2pt]{};
			\draw (2.5,.5) node[fill,circle,inner sep=2pt]{};
		}
		
		\onslide<2>{
			\draw [color=blue] (.25,.25) node[fill,circle,inner sep=1.5pt]{};
			\draw [color=blue] (.25,.75) node[fill,circle,inner sep=1.5pt]{};
			\draw [color=blue] (.75,.25) node[fill,circle,inner sep=1.5pt]{};
			\draw [color=blue] (.75,.75) node[fill,circle,inner sep=1.5pt]{};
			\draw [color=blue] (1,1.5) node[fill,circle,inner sep=1.5pt]{};
			\draw [color=blue] (1.5,0) node[fill,circle,inner sep=1.5pt]{};
			\draw [color=blue] (1.5,1) node[fill,circle,inner sep=1.5pt]{};
			\draw [color=blue] (2.25,-.25) node[fill,circle,inner sep=1.5pt]{};
			\draw [color=blue] (2.25,.25) node[fill,circle,inner sep=1.5pt]{};
		}
		
		\draw<3-5> [color=red, very thick] (1,0) -- (2,0);
		
		\draw<3> [color=red] (1.5,0) node[below,align=center]{$e$};
		\draw<4-5> [color=red] (1.5,0) node[below,align=center]{$[e]_\parallel$};
		\draw<5> [color=blue] (1.5,0) node[fill,circle,inner sep=1.5pt]{};
		\draw<5-> [color=blue] (1.5,0) node[above,align=center]{$H(e)$};
		
		\onslide<6->{
			\draw [color=purple] (1,2) -- (1,1);
			\draw [color=purple] (2,1) -- (1,1) -- (0,0);
			\draw [color=purple] (0,1) -- (1,0) -- (1.5,0);
			\draw [color=teal] (1.5,0) -- (2,0) -- (2.5,.5);
			\draw [color=teal] (2,0) -- (2.5,-.5);
			
			\draw [color=purple] (0,0) node[fill,circle,inner sep=2pt]{};
			\draw [color=purple] (.5,.5) node[fill,circle,inner sep=2pt]{};
			\draw [color=purple] (0,1) node[fill,circle,inner sep=2pt]{};
			\draw [color=purple] (1,1) node[fill,circle,inner sep=2pt]{};
			\draw [color=purple] (1,2) node[fill,circle,inner sep=2pt]{};
			\draw [color=purple] (2,1) node[fill,circle,inner sep=2pt]{};
			\draw [color=purple] (1,0) node[fill,circle,inner sep=2pt]{};
			\draw [color=teal] (2,0) node[fill,circle,inner sep=2pt]{};
			\draw [color=teal] (2.5,-.5) node[fill,circle,inner sep=2pt]{};
			\draw [color=teal] (2.5,.5) node[fill,circle,inner sep=2pt]{};
			\draw [color=blue] (1.5,0) node[fill,circle,inner sep=1.5pt]{};
			
			\draw [color=purple] (.5,.7) node[above]{$U^-$};
			\draw [color=teal] (2.2,0) node[right]{$U^+$};			
		}
	\end{tikzpicture} \hspace{1cm}
	\begin{tikzpicture}[scale=1.2]
			\onslide<7->{
				\draw (0,4) node[right]{$x_1$};
				\draw (4,0) node[above]{$x_2$};
			}
			
			\onslide<7-11>{
				\draw [->,ultra thick] (-1.5,0) -- (4,0);
				\draw [->,ultra thick] (0,-1.5) -- (0,4);
				\foreach \x in {-1,1,2,3} {
					\draw (-1.5,\x) -- (3.5,\x);
					\draw (\x,-1.5) -- (\x,3.5);
				}
			}
			
			\onslide<8>{
				\draw [color=red, very thick] (1,1) -- (2,1);
				\draw [color=red] (2,1) node[align=left,anchor=north west]{$e$};
			}
			
			\onslide<9-11>{
				\foreach \x in {-1,0,...,3} {
					\draw [very thick, color=red] (1,\x) -- (2,\x);
				}
				\draw [color=red] (2,1) node[align=left,anchor=north west]{$[e]_\parallel$};
			}
			
			\onslide<10>{
				\foreach \x in {0,...,3} {
					\draw [dashed, very thick, color=blue] (\x-.5,-1.5) -- (\x-.5,3.5);
					\draw [dashed, very thick, color=blue] (-1.5,\x-.5) -- (3.5,\x-.5);
				}
			}
			
			\onslide<11>{
				\draw [very thick, color=blue] (1.5,3.5) -- (1.5,-1.5) node[below]{$H(e)$};
			}
			
			\onslide<12>{
				\path [schraffiert=purple] (-1.5,-1.5) rectangle (1.5,3.5);
				\path [schraffiert=teal] (1.5,3.5) rectangle (3.5,-1.5);
				
				\draw [color=purple] (-.5,-1.5) node[below]{$U^-$};
				\draw [color=teal] (3,-1.5) node[below]{$U^+$};
				
				\draw [->,ultra thick] (-1.5,0) -- (4,0);
				\draw [->,ultra thick] (0,-1.5) -- (0,4);
				\foreach \x in {-1,1,2,3} {
					\draw (-1.5,\x) -- (3.5,\x);
					\draw (\x,-1.5) -- (\x,3.5);
				}
				
				\draw [color=red, ultra thick] (1,1) -- (2,1);
				\draw [color=red] (2,1) node[align=left,anchor=north west]{$e$};
				
				\draw [very thick, color=blue] (1.5,3.5) -- (1.5,-1.5) node[below]{$H(e)$};
			}
	\end{tikzpicture}
\end{frame}

\begin{frame}[c]{Mehr über kubische Komplexe}
	$(X,d)$: wegzusammenhängender kubischer Komplex
	\begin{itemize}
		\item Wann ist $(X,d)$ vollständig? \\
			$\rightarrow$ wenn $(X,d)$ endlich dimensional oder lokal endlich ist.
		\item Wann ist $(X,d)$ $\cat$? \\
			$\rightarrow$ Gromov's link condition
	\end{itemize}
	\vspace*{2cm}
	Mehr dazu: \\
	\qquad P. Schwer: Lecture Notes on $\cat$ Cubical Complexes
\end{frame}
\end{document}