%!TEX root = ../Article.tex
% -- Author: Phil Steinhorst, p.st@wwu.de
\thispagestyle{empty}
\begin{center}
\begin{minipage}{0.4\textwidth}
\begin{flushleft}
\includegraphics[height=1.5cm,keepaspectratio]{../img/wwulogo.pdf}
\end{flushleft}
\end{minipage}
\hfill
\begin{minipage}{0.4\textwidth}
\begin{flushright}
\vspace*{0.3cm}
\includegraphics[height=1.2cm,keepaspectratio]{../img/fb10logo.pdf} \
\end{flushright}
\end{minipage}

\vspace*{2cm}
\textbf{\LARGE \vortrag} \\
\vspace{0.6cm}
{\LARGE \verfasser} \\
{\normalsize \texttt{p.st@wwu.de}} \\
\vspace{0.6cm} 
{\Large Seminar zur Gruppentheorie und Geometrie: \\ Kazhdan- und Haagerup-Eigenschaften von Gruppen} \\
\vspace{0.6cm}
\Large{\semester}
\end{center}

\section*{Einleitung}
	In den bisherigen Vorträgen des Seminars wurde der Begriff des $\cat$-Raumes sowie die Kazhdan-Eigenschaft $\prT$ für Gruppen vorgestellt. Dieser Vortrag beschäftigt sich mit einem Resultat über Gruppen, die die Eigenschaft $\prT$ erfüllen und ihre Wirkung auf spezielle kubische Komplexe. 
	
\begin{satz2}
	Sei $X$ ein vollständiger $\cat$ kubischer Komplex und $G$ eine endlich erzeugte Gruppe, die die Kazhdan-Eigenschaft $\prT$ erfüllt. Dann hat jede simpliziale Wirkung $\Phi$ von $G$ auf $X$ einen globalen Fixpunkt, das heißt es existiert ein $x \in X$, sodass $\Phi(g)(x) = x$ für alle $g \in G$.
\end{satz2}